    \usepackage{graphicx} % Used to insert images
	\usepackage{float} % Allows figures to be locked
    \usepackage[table,xcdraw]{xcolor}
    \usepackage{xfrac}
    \usepackage{rotating}
    \usepackage{adjustbox} % Used to constrain images to a maximum size
    \usepackage{color} % Allow colors to be defined
    \usepackage{enumerate} % Needed for markdown enumerations to work
    \usepackage{enumitem} % Needed for lettered enumeration
    \usepackage{geometry} % Used to adjust the document margins
    \usepackage{amsmath} % Equations
    \usepackage{amssymb} % Equations
    \usepackage{eurosym} % defines \euro
    \usepackage[mathletters]{ucs} % Extended unicode (utf-8) support
    \usepackage[utf8x]{inputenc} % Allow utf-8 characters in the tex document
    \usepackage{fancyvrb} % verbatim replacement that allows latex
    \usepackage{grffile} % extends the file name processing of package graphics
                         % to support a larger range
    \usepackage{longtable} % longtable support required by pandoc >1.10
    \usepackage{booktabs}  % table support for pandoc > 1.12.2
    \usepackage{amsthm}
	\usepackage{pdfpages} % importing of PDFs
    \usepackage{ifthen}
    \usepackage[nomessages]{fp}
	
	\usepackage[english]{babel}
	\usepackage{lipsum} % Random text
	\usepackage{array}
	\usepackage{multirow}
	\usepackage{mathtools}

    \usepackage{tikz} % Needed to box output/input
    \usepackage{scrextend} % Used to indent output
    \usepackage{needspace} % Make prompts follow contents
    \usepackage{framed} % Used to draw output that spans multiple pages

    \usepackage{fancyhdr}
    \usepackage{lastpage}
    \renewcommand{\headrulewidth}{0.4pt}

    \newcommand{\expnumber}[2]{{#1}\mathrm{e}{#2}}

    \definecolor{orange}{cmyk}{0,0.4,0.8,0.2}
    \definecolor{darkorange}{rgb}{.71,0.21,0.01}
    \definecolor{darkgreen}{rgb}{.12,.54,.11}
    \definecolor{myteal}{rgb}{.26, .44, .56}
    \definecolor{gray}{gray}{0.45}
    \definecolor{lightgray}{gray}{.95}
    \definecolor{mediumgray}{gray}{.8}
    \definecolor{inputbackground}{rgb}{.95, .95, .85}
    \definecolor{outputbackground}{rgb}{.95, .95, .95}
    \definecolor{traceback}{rgb}{1, .95, .95}
    % ansi colors
    \definecolor{red}{rgb}{.6,0,0}
    \definecolor{green}{rgb}{0,.65,0}
    \definecolor{brown}{rgb}{0.6,0.6,0}
    \definecolor{blue}{rgb}{0,.145,.698}
    \definecolor{purple}{rgb}{.698,.145,.698}
    \definecolor{cyan}{rgb}{0,.698,.698}
    \definecolor{lightgray}{gray}{0.5}

    % bright ansi colors
    \definecolor{darkgray}{gray}{0.25}
    \definecolor{lightred}{rgb}{1.0,0.39,0.28}
    \definecolor{lightgreen}{rgb}{0.48,0.99,0.0}
    \definecolor{lightblue}{rgb}{0.53,0.81,0.92}
    \definecolor{lightpurple}{rgb}{0.87,0.63,0.87}
    \definecolor{lightcyan}{rgb}{0.5,1.0,0.83}

\usepackage[colorlinks=true,
			breaklinks=true,  % so long urls are correctly broken across lines
			urlcolor=blue,
			linkcolor=darkorange,
			citecolor=darkgreen,
			anchorcolor=blue]{hyperref}
			
    \usepackage{fontspec}
    \usepackage{xunicode}
    \usepackage{titlesec}
    \defaultfontfeatures{Ligatures=TeX}
    \setmainfont{Cambria}
    \newfontfamily\secfont{Calibri}
    \newcommand{\chapstyle}{\bfseries\Huge\secfont\color{red}}
    \newcommand{\secstyle}{\bfseries\Large\secfont\color{red}}
    \newcommand{\subsecstyle}{\bfseries\large\secfont\color{red}}
    \newcommand{\subsubsecstyle}{\itshape\large\secfont\color{red}}
    \titleformat{\chapter}{\chapstyle}{}{.1ex}{\hspace{-.4pt}\Huge
      Chapter \thechapter\newline}[\normalfont]
    \titleformat{\section}{\secstyle}{}{.1ex}{\hspace{-.4pt}\Large\S\thesection\hspace{0.6em}}[\normalfont]
    \titleformat{\subsection}{\subsecstyle}{}{.1ex}{\hspace{-.4pt}\large\S\thesubsection\hspace{0.6em}}[\normalfont]
    \titleformat{\subsubsection}{\subsubsecstyle}{}{.1ex}{\hspace{-.4pt}\itshape\large\S\thesubsubsection\hspace{0.6em}}[\normalfont]

    \titleformat{name=\section,numberless}{\secstyle}{}{.1ex}{\hspace{-8pt}\Large\hspace{0.6em}}[\normalfont]
    \titleformat{name=\subsection,numberless}{\subsecstyle}{}{.1ex}{\hspace{-8pt}\large\hspace{0.6em}}[\normalfont]
    \titleformat{name=\subsubsection,numberless}{\subsubsecstyle}{}{.1ex}{\hspace{-8pt}\itshape\large\hspace{0.6em}}[\normalfont]


\newtheorem{remark}{Remark}
%\newtheorem*{hint}{{\color{red}Hint}}
\newenvironment*{hint}%
{ \noindent\ignorespaces{\color{red}\textbf{Hint:}} }%
  { \par\noindent\ignorespacesafterend }

    % Prevent overflowing lines due to hard-to-break entities
    \sloppy
    % The hyperref package gives us a pdf with properly built
    % internal navigation ('pdf bookmarks' for the table of contents,
    % internal cross-reference links, web links for URLs, etc.)
    \usepackage[colorlinks=true,
                breaklinks=true,  % so long urls are correctly broken across lines
                urlcolor=blue,
                linkcolor=darkorange,
                citecolor=darkgreen,
                anchorcolor=blue]{hyperref}
    % Slightly bigger margins than the latex defaults

    \geometry{verbose,tmargin=1in,bmargin=1in,lmargin=1in,rmargin=1in}

    %\setlength{\parindent}{0pt}
    \usepackage{parskip}

\usepackage{boxedminipage}

\usepackage{calc}
\newlength{\remdatea}
\newlength{\remdateb}
\newlength{\remnamea}
\newlength{\remnameb}
\newcommand{\ifem}{\texttt{IFEM}}



% --- mint code highlighting
\usepackage[newfloat]{minted}
\usepackage{color}
\SetupFloatingEnvironment{listing}{name=Listing}
\usepackage{tcolorbox}
\usepackage{mdframed}
\usepackage{etoolbox}
\definecolor{bg}{rgb}{.95,.95,.95}
\BeforeBeginEnvironment{minted}{\medskip\begin{mdframed}[backgroundcolor=bg]}
\AfterEndEnvironment{minted}{\end{mdframed}\par\medskip}
\newcommand{\mintit}[1]{\mintinline{python}{#1}}
\newcommand{\mintitc}[1]{\mintinline{c}{#1}}
\newcommand{\mintitm}[1]{\mintinline{matlab}{#1}}

\newmintedfile[pycode]{python}{
bgcolor=bg,
fontfamily=tt,
fontsize=\scriptsize,
linenos=true,
numberblanklines=true,
numbersep=5pt,
gobble=0,
frame=leftline,
framerule=0.4pt,
framesep=2mm,
funcnamehighlighting=true,
tabsize=4,
obeytabs=false,
mathescape=false
samepage=false, %with this setting you can force the list to appear on the same page
showspaces=false,
showtabs =false,
texcl=false,
}

\newmintedfile[mcode]{matlab}{
bgcolor=bg,
fontfamily=tt,
fontsize=\scriptsize,
linenos=true,
numberblanklines=true,
numbersep=5pt,
gobble=0,
frame=leftline,
framerule=0.4pt,
framesep=2mm,
funcnamehighlighting=true,
tabsize=4,
obeytabs=false,
mathescape=false
samepage=false, %with this setting you can force the list to appear on the same page
showspaces=false,
showtabs =false,
texcl=false,
}


%%% ====================================================================
%%% Custom code highlighting using Minted and tikz
%%% ====================================================================

\usepackage{tikz}

%%% ====================================================================
%%% Specific colors used for code highlighting
%%% ====================================================================

% Matlab specific colors
\definecolor{matlabcodeBG}{rgb}{0.99,0.99,1} % light blue
\definecolor{matlabcodeCOL}{RGB}{34,43,53}
\definecolor{matlabcodeCOLNUM}{RGB}{128,128,128}

% Python specific colors
\definecolor{pythoncodeBG}{rgb}{0.99,1,0.99} % light green
\definecolor{pythoncodeTCBborder}{RGB}{94,94,94} % dark gray
\definecolor{pythoncodeCOL}{RGB}{226,226,226} % light gray
\definecolor{pythoncodeCOLNUM}{RGB}{0,0,0} % black

% Python spyder dark color profile
%\definecolor{pythoncodeBG}{RGB}{38,50,56} % dark
%\definecolor{pythoncodeTCBborder}{RGB}{226,226,226} % light gray
%\definecolor{pythoncodeCOL}{RGB}{34,43,53} % dark
%\definecolor{pythoncodeCOLNUM}{RGB}{128,128,128} % gray

% commands for color
\newcommand{\bgcol}{\color{pythoncodeBG}}
\newcommand{\bgcolCol}{\color{pythoncodeCOL}}
\newcommand{\bgcolNum}{\color{pythoncodeCOLNUM}}

% TCB colorbox to put higlighted minted code from pygments inside
\usepackage{tcolorbox}
\tcbuselibrary{minted,skins,breakable}

\usepackage{minted}

% Different languages
\newcommand{\python}{python}
\newcommand{\matlab}{matlab}
\newcommand{\octave}{octave}

% Different styles
\newcommand{\codeStyle}{default}
%\newcommand{\codeStyle}{material}
%\newcommand{\codeStyle}{perldoc}

\usemintedstyle[python]{\codeStyle} % Specific color scheme
\usemintedstyle[matlab]{\codeStyle} % Specific color scheme
\usemintedstyle[octave]{\codeStyle} % Specific color scheme

%%% ====================================================================
%%% Custom code highlighting using Minted
%%% ====================================================================

\newcommand{\pyInputfns}{\pythoninputfns}
%\newcommand{\codeStyle}{material}

% Python normalsize
\newtcbinputlisting{\pythoninput}[2][]{%
	listing file={#2},
	minted language=python,
	minted style=default,
	minted options={
		fontsize=\normal pythoncodeCOLNUM,
		linenos,
		numbersep=1mm,
		breaklines=true,
	},% <-- put other minted options inside the brackets
	overlay={%
		\begin{tcbclipinterior}
			\fill[gray!25] (frame.south west) rectangle ([xshift=5mm]frame.north west);
		\end{tcbclipinterior}
	},
	colback=pythoncodeBG,
	colframe=black!70,
	before skip=5pt plus 2pt,
	breakable,
	enhanced,% <-- put other tcolorbox options here
	listing only,#1
}
% Python footnotesize
\newtcbinputlisting{\pythoninputfns}[2][]{%
	listing file={#2},
	minted language=\python,
	minted style=\codeStyle,
	minted options={
		fontsize=\footnotesize\bgcolNum,
		linenos,
		numbersep=1mm,
		breaklines=true,
	},% <-- put other minted options inside the brackets
	overlay={%
		\begin{tcbclipinterior}
			\fill[pythoncodeCOL] (frame.south west) rectangle ([xshift=5mm]frame.north west); % gray!25
		\end{tcbclipinterior}
	},
	colback=pythoncodeBG,
	colframe=pythoncodeTCBborder, %colframe=black!70,
	before skip=5pt plus 2pt,
	breakable,
	enhanced,% <-- put other tcolorbox options here
	listing only,#1
}
% Python tiny
\newtcbinputlisting{\pythoninputtiny}[2][]{%
	listing file={#2},
	minted language=python,
	minted style=default,
	minted options={
		fontsize=\tiny,
		linenos,
		numbersep=1mm,
		breaklines=true,
	},% <-- put other minted options inside the brackets
	overlay={%
		\begin{tcbclipinterior}
			\fill[gray!25] (frame.south west) rectangle ([xshift=5mm]frame.north west);
		\end{tcbclipinterior}
	},
	colback=pythoncodeBG,
	colframe=black!70,
	before skip=5pt plus 2pt,
	breakable,
	enhanced,% <-- put other tcolorbox options here
	listing only,#1
}
% Matlab normal
\newtcbinputlisting{\matlabinput}[2][]{%
	listing file={#2},
	minted language=matlab,
	minted style=default,
	minted options={
		fontsize=\normal,
		linenos,
		numbersep=1mm,
		breaklines=true,
	},% <-- put other minted options inside the brackets
	overlay={%
		\begin{tcbclipinterior}
			\fill[gray!25] (frame.south west) rectangle ([xshift=5mm]frame.north west);
		\end{tcbclipinterior}
	},
	colback=matlabcodeBG,
	colframe=black!70,
	before skip=5pt plus 2pt,
	breakable,
	enhanced,% <-- put other tcolorbox options here
	listing only,#1
}
% Matlab footnotesize
\newtcbinputlisting{\matlabinputfns}[2][]{%
	listing file={#2},
	minted language=matlab,
	minted style=default,
	minted options={
		fontsize=\footnotesize,
		linenos,
		numbersep=1mm,
		breaklines=true,
	},% <-- put other minted options inside the brackets
	overlay={%
		\begin{tcbclipinterior}
			\fill[gray!25] (frame.south west) rectangle ([xshift=5mm]frame.north west);
		\end{tcbclipinterior}
	},
	colback=matlabcodeBG,
	colframe=black!70,
	before skip=5pt plus 2pt,
	breakable,
	enhanced,% <-- put other tcolorbox options here
	listing only,#1
}
% Matlab tiny
\newtcbinputlisting{\matlabinputtiny}[2][]{%
	listing file={#2},
	minted language=matlab,
	minted style=default,
	minted options={
		fontsize=\tiny,
		linenos,
		numbersep=1mm,
		breaklines=true,
	},% <-- put other minted options inside the brackets
	overlay={%
		\begin{tcbclipinterior}
			\fill[gray!25] (frame.south west) rectangle ([xshift=5mm]frame.north west);
		\end{tcbclipinterior}
	},
	colback=matlabcodeBG,
	colframe=black!70,
	before skip=5pt plus 2pt,
	breakable,
	enhanced,% <-- put other tcolorbox options here
	listing only,#1
}

%%%%%%%%%%%%%%%%%%%%%%%%%  Function used to generate vectors and tensors %%%%%%%%%
\usepackage{stackengine}
\stackMath
\newcommand\tensor[2][1]{%
\def\useanchorwidth{T}%
	\ifnum#1>1%
		\stackunder[0pt]{\tensor[\numexpr#1-1\relax]{#2}}{\scriptscriptstyle \sim}%
	\else%
		\stackunder[1pt]{#2}{\scriptscriptstyle \sim}%
\fi%
}
%%%%%%%%%%%%%%%%%%%%%

%%% Local Variables:
%%% mode: latex
%%% TeX-master: t
%%% End:
