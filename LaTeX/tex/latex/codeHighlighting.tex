%%% ====================================================================
%%% Custom code highlighting using Minted and tikz
%%% ====================================================================

\usepackage{tikz}


%%% ====================================================================
%%% Specific colors used for code highlighting
%%% ====================================================================

\definecolor{matlabcodebg}{rgb}{0.99,0.99,1}
\definecolor{pythoncodebg}{rgb}{0.99,1,0.99}
\definecolor{ccodebg}{rgb}{1,0.99,0.99}
\definecolor{fortrancodebg}{rgb}{1,1,0.99}
\definecolor{latexcodebg}{HTML}{FFFBF1}

%\usepackage{minted} % Already charged with tcblibrary minted

\usepackage{tcolorbox}
\tcbuselibrary{minted,skins,breakable,hooks}

%%% ====================================================================
%%% Custom code highlighting using Minted
%%% ====================================================================

\usepackage{fontawesome} % Additional features such as symbols
\newcolumntype{\CeX}{>{\centering\let\newline\\\arraybackslash}X}%
\newcommand{\TwoSymbolsAndText}[3]{%
  \begin{tabularx}{\textwidth}{c\CeX c}%
    #1 & #2 & #3
  \end{tabularx}%
}

\newcounter{code} % Number input code files
\crefname{code}{Script}{Scripts} % Used for referencing with \cref{code:...}

% The "\codeFromFile" function is used in the following manner:
% \codeFromFile
% {language}
% {\subfix{path}}
% {Header}
% {label}
% {fontsize}
% {backgroundcolor}
% {mintedStyle}

\newtcbinputlisting[use counter=code, number format=\arabic]{\codeFromFile}[7]{%
    listing engine=minted, 
    minted language=#1, 
    minted style={#7}, 
    listing file={#2}, 
    minted options={
        fontsize=#5, 
        fontfamily=cmtt,  % default
        fontseries=m, % default
        fontshape=n, % default
        linenos=True, 
        xleftmargin=3.25mm, 
        numbersep=2mm, 
        stepnumber=1, 
        numberblanklines=True, 
        breaklines=true, 
        },% <-- put other minted options inside the brackets
    overlay={%
        \begin{tcbclipinterior}
            \fill[gray!25] (frame.south west) rectangle ([xshift=5mm]frame.north west);
        \end{tcbclipinterior}
        },
    colback=#6, 
    colframe=black!70, 
    coltitle=white, 
    coltext=black, 
    before skip= \UofUHeadSpace, 
    after skip= \UofUHeadSpace, 
    code={\singlespacing}, 
    breakable, % Allows for code to continue on multiple pages
    listing only, 
    arc=1.5mm, % Curvature of line corner
    enhanced jigsaw, % ,
    boxrule=0.5mm, % Border width
    size=title,
    title=\TwoSymbolsAndText{\faCode}{%
    \textbf{Script \thetcbcounter}\ifthenelse{\equal{#3}{}}{}{\textbf{:} \textit{#3}}%
    }{\faCode}, % Name of each program heading box.  
    label=code:#4 % Label used to refer to code
}
